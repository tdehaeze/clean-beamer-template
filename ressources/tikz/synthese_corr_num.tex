\documentclass[12pt,tikz]{standalone}

\ifstandalone%
    \usepackage{import}
    \import{../../configuration/}{comon_packages.tex}%
    \import{../../configuration/}{variables.tex}%
    \import{../../configuration/}{conftikz.tex}%
    \import{../../configuration/}{custom_config.tex}%
\fi

\begin{document}
    \begin{tikzpicture}[.style={font issue=\footnotesize}]
        \node[draw, align=center, text width=2cm] (m_continu) at (0, 0) {Modèle continu};
        \node[draw, align=center, text width=2cm] (m_discret) at (6, 0) {Modèle discret};
        \node[draw, align=center, text width=2cm] (m_continu_eq) at (12, 0) {Modèle continu équivalent};
        \node[draw, align=center, text width=2cm] (c_continu_eq) at (12, -4) {Correcteur continu équivalent};
        \node[draw, align=center, text width=2cm] (c_discret) at (6, -4) {Correcteur discret};

        \draw[->, >= latex, dashed, postaction={decorate,decoration={raise=1ex,text along path,text align=center,text={Synthese de correcteur}}}, postaction={decorate,decoration={raise=-2.5ex,text along path,text align=center,text={numerique}}}] (m_continu) to[bend right] (c_discret);
        \draw[->, >= latex] (m_continu) -- node[above]{Discrétisation}node[below]{bloqueur ordre 0} (m_discret);
        \draw[->, >= latex] (m_discret) -- node[above]{Transformation}node[below]{bilinéaire} (m_continu_eq);
        \draw[->, >= latex] (m_continu_eq) -- node[left, align=center, text width=2cm]{Synthèse continue} (c_continu_eq);
        \draw[->, >= latex] (c_continu_eq) -- node[above]{Transformation}node[below]{inverse} (c_discret);
    \end{tikzpicture}
\end{document}

