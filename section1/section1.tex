\documentclass[../main/main.tex]{subfiles}

\begin{document}

\onlyinsubfile{%
}

% ==================================================================
% Section : Introduction
% ==================================================================
\section{Introduction}%
\label{sec:introduction}

\begin{frame}{Introduction - Outline}
  \tableofcontents[currentsection, hideothersubsections]
\end{frame}

% ==================================================================
\subsection{Classical and Modern control}
\begin{frame}[t]{Classical and Modern Control}
  \begin{columns}
    \begin{column}{0.5\textwidth}
      \stitle{Classical control (1930)}\\\vspace{1em}
      TF (input-output)\\
      Nyquist, Bode, Black, \ldots\\
      P-PI-PID, Phase lead-lag, \ldots\vspace{1em}
      \begin{tcolorbox}[size=small, top=4pt, colback=green!5!white,colframe=green!75!black,title=Advantages]
        \begin{itemize}
        \item Stability
        \item Performances
        \item Robustness % TODO pas trop compris pourquoi
        \end{itemize}
      \end{tcolorbox}\vspace{1em}
      \begin{tcolorbox}[size=small, colback=red!5!white,colframe=red!75!black,title=Disadvantages]
        \begin{itemize}
        \item Mannual methods
        \item SISO
        \end{itemize}
      \end{tcolorbox}
    \end{column}
    \begin{column}{0.5\textwidth}
      \stitle{Modern Control (1960)}\\\vspace{1em}
      State space representation\\
      Optimal command\\
      LQR-LQG\vspace{1em}
      \begin{tcolorbox}[size=small, colback=green!5!white,colframe=green!75!black,title=Advantages]
        \begin{itemize}
        \item Automatic synthesis
        \item MIMO
        \item Optimisation problem
        \end{itemize}
      \end{tcolorbox}\vspace{1em}
      \begin{tcolorbox}[size=small, colback=red!5!white,colframe=red!75!black,title=Disadvantages]
        \begin{itemize}
        \item Robustness
        \item Perturbation rejection
        \end{itemize}
      \end{tcolorbox}
    \end{column}
  \end{columns}
\end{frame}
% ==================================================================


% ==================================================================
\subsection{H-infinity Synthesis}
\begin{frame}[t]{\(\hinf\) Synthesis (1981 \(\rightarrow\) 1994)}
  \begin{tcolorbox}[size=small, top=4pt, colback=green!5!white,colframe=green!75!black,title=Advantages]
    \vspace{-1em}\begin{columns}
      \begin{column}{0.5\textwidth}
        \begin{itemize}
        \item Stability
        \item Performances
        \item Robustness % TODO pas trop compris pourquoi
        \end{itemize}
      \end{column}
      \begin{column}{0.5\textwidth}
        \begin{itemize}
        \item Automatic synthesis
        \item MIMO
        \item Optimisation problem
        \end{itemize}
      \end{column}
    \end{columns}
  \end{tcolorbox}
  \begin{tcolorbox}[size=small, top=4pt, colback=red!5!white,colframe=red!75!black,title=Disadvantages]
    \begin{itemize}
    \item Need for a reasonably good model of the system
    \end{itemize}
  \end{tcolorbox}
  \begin{tcolorbox}[size=small, colback=blue!5!white,colframe=blue!75!black,title=Applications]
    \begin{itemize}
    \item \(\hinf\) control for the ARIANE 5 plus launcher
    \item Vibration Control (Active suspension in cars)
    \item Microelectronics: micromachined sensors
    \end{itemize}
  \end{tcolorbox}
\end{frame}


\begin{frame}{\(\hinf\) Syntesis - The philosophy}
  \begin{overprint}
    \onslide<1>\begin{tcolorbox}[size=small,
      top=4pt,
      halign=center,
      colback=blue!5!white,
      colframe=blue!50!black,
      subtitle style={boxrule=0.4pt,
        colback=blue!50!black},
      title=\(\hinf\) Synthesis - Hypothesis]
      Linear Time Invariant (LTI) models\\
      Continous systems
      \tcbsubtitle{It works in the frequency domain}
      We have to \textbf{translate the specifications in the frequency domain}
      \tcbsubtitle{It works on the \textbf{closed loop} tranfert functions}
      It uses an algorithm to find a controller (\emph{if it exists}) so that the CL TF have the behaviour that we want
      \tcbsubtitle{Selection of the weighting functions}
      We have to determine the ''\emph{shapes}'' or the CL TF so that the behaviour of the system matches the specifications
    \end{tcolorbox}
    \onslide<2>\begin{tcolorbox}[size=small,
      top=4pt,
      halign=center,
      colback=blue!5!white,
      colframe=blue!50!black,
      subtitle style={boxrule=0.4pt,
        colback=blue!50!black},
      title=\(\hinf\) Synthesis]
      \begin{itemize}
      \item Computer-Aided Control Systems Design
      \item At the end, \(\hinf\) synthesis gives you a controller that can be as
        simple as a PI or PID
      \item Then we should analyse what is the controller obtained (leads, lags, \ldots)
      \end{itemize}
      \tcbsubtitle{Challenging part}
      \begin{itemize}
      \item Transform the problem into a standard \(\hinf\) problem
      \item Translate the specifications into weighting functions
      \end{itemize}
    \end{tcolorbox}
  \end{overprint}
\end{frame}
% ==================================================================

\end{document}

