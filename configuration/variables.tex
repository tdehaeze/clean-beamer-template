% =============================================================
% For the title page
% =============================================================
% Report
\newcommand{\reportTitle}{Clean Beamer Template}
\newcommand{\reportSubject}{Clean Beamer Template}
\newcommand{\reportType}{How to use it?}
\newcommand{\reportShortName}{TFE 2017}
\newcommand{\reportDates}{\today}
\newcommand{\reportAuthor}{Dehaeze Thomas}

% Author
\newcommand{\authorFirstName}{Thomas}
\newcommand{\authorLastName}{Dehaeze}
\newcommand{\authorEmail}{dehaeze.thomas@gmail.com}

% Company One
\newcommand{\companyOneName}{ESRF}
\newcommand{\companyOnePlace}{Grenoble}
\newcommand{\companyOneLogo}{logo-esrf.png}

% Person One
\newcommand{\personOnePosition}{Supervisor}
\newcommand{\personOneFirstName}{Surname}
\newcommand{\personOneLastName}{Name}
\newcommand{\personOneEmail}{surname.name@domain.com}

% Company Two
\newcommand{\companyTwoName}{LTAS}
\newcommand{\companyTwoPlace}{Liège}
\newcommand{\companyTwoLogo}{logo-ltas.png}

% Person Two
\newcommand{\personTwoPosition}{Supervisor}
\newcommand{\personTwoFirstName}{Surname}
\newcommand{\personTwoLastName}{Name}
\newcommand{\personTwoEmail}{name.surname@domain.com}

\newcommand{\topLogo}{logo-ltas}
% =============================================================


% =============================================================
% Other variables
% =============================================================
% changes the default title for the nomenclature
\newcommand{\nomenlaturename}{Nomenclature}
\newcommand{\bibliographyname}{Bibliography}
\newcommand{\listoffiguresname}{List of figures}
\newcommand{\listingscaptionname}{Source code}
\newcommand{\listoflistingsname}{List of source code}
% =============================================================


% =============================================================
% Path
% =============================================================
% Graphic Path
\graphicspath{%
    {../ressources/}%
    {../ressources/pdf/}%
    {../ressources/images/}%
    {../ressources/logos/}%
    {../ressources/tikz/}%
}

% Source code Path
\newcommand{\codefolder}[1]{../ressources/code/#1}
% =============================================================

% =============================================================
% Colors
% =============================================================
\definecolor{colorblack}{rgb}{0, 0, 0}
\definecolor{colorblue}{rgb}{0, 0.4470, 0.7410}
\definecolor{colorred}{rgb}{0.8500, 0.3250, 0.0980}
\definecolor{coloryellow}{rgb}{0.9290, 0.6940, 0.1250}
\definecolor{colorpurple}{rgb}{0.4940, 0.1840, 0.5560}
\definecolor{colorgreen}{rgb}{0.4660, 0.6740, 0.1880}
\definecolor{colorcyan}{rgb}{0.3010, 0.7450, 0.9330}
\definecolor{colorbordeau}{rgb}{0.6350, 0.0780, 0.1840}

% Main color
\definecolor{maincolor}{RGB}{89, 9, 38}
\definecolor{secondcolor}{RGB}{20, 9, 89}

\definecolor{LightGray}{rgb}{0.45,0.5,0.45}
% =============================================================


% =============================================================
% Some variable to customize theme
% =============================================================
% toogletrue to have a "fancy chapter" tooglefalse to don't
\newtoggle{fancychapter}
\toggletrue{fancychapter}

% toogletrue to put section numbering into margin tooglefalse to don't
\newtoggle{sectionmargin}
\toggletrue{sectionmargin}

% toogletrue to have a mini toc for each chapter tooglefalse to don't
\newtoggle{minitocchapter}
\toggletrue{minitocchapter}

% toogletrue to print "Confidentiel"
\newtoggle{isconfidential}
\toggletrue{isconfidential}
\newcommand{\confidential}{\color{maincolor}\textsc{confidential}}
% =============================================================


% =============================================================
% Some usefull commands
% =============================================================
\newcommand{\customvrule}[0]{%
    \begin{column}{0.02\textwidth}%
        \includegraphics[height=0.9\textheight, keepaspectratio]{vert_line}%
    \end{column}%
}

\newcommand{\customhrule}[0]{%
    \includegraphics[width=\linewidth, keepaspectratio]{horiz_line}\hfill%
}

\newcommand{\ccaption}[1]{%
    \\[2pt]{\centering\color{mycolor2}\small\textit{\underline{#1}}}
}

\newcommand{\cimportant}[1]{%
    {\centering\color{mycolor2}\textbf{#1}}
}
% =============================================================

